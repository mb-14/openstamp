% This must be in the first 5 lines to tell arXiv to use pdfLaTeX, which is strongly recommended.
\pdfoutput=1
% In particular, the hyperref package requires pdfLaTeX in order to break URLs across lines.

\documentclass[11pt]{article}

% Change "review" to "final" to generate the final (sometimes called camera-ready) version.
% Change to "preprint" to generate a non-anonymous version with page numbers.
\usepackage[review]{acl}

% Standard package includes
\usepackage{times}
\usepackage{latexsym}
\usepackage{amssymb}
\usepackage{amsmath}
\usepackage{amsthm}
\newtheorem{theorem}{Theorem}
\setcounter{secnumdepth}{3}


% For proper rendering and hyphenation of words containing Latin characters (including in bib files)
\usepackage[T1]{fontenc}
% For Vietnamese characters
% \usepackage[T5]{fontenc}
% See https://www.latex-project.org/help/documentation/encguide.pdf for other character sets

% This assumes your files are encoded as UTF8
\usepackage[utf8]{inputenc}

% This is not strictly necessary, and may be commented out,
% but it will improve the layout of the manuscript,
% and will typically save some space.
\usepackage{microtype}

% This is also not strictly necessary, and may be commented out.
% However, it will improve the aesthetics of text in
% the typewriter font.
\usepackage{inconsolata}

%Including images in your LaTeX document requires adding
%additional package(s)
\usepackage{graphicx}

% If the title and author information does not fit in the area allocated, uncomment the following
%
%\setlength\titlebox{<dim>}
%
% and set <dim> to something 5cm or larger.
\newcommand{\sr}[1]{\textcolor{purple}{\textbf{\small [SR: #1]}}}

\title{Watermarking Open-Source LLMs with Context-Aware Logit Biasing}

% Author information can be set in various styles:
% For several authors from the same institution:
% \author{Author 1 \and ... \and Author n \\
%         Address line \\ ... \\ Address line}
% if the names do not fit well on one line use
%         Author 1 \\ {\bf Author 2} \\ ... \\ {\bf Author n} \\
% For authors from different institutions:
% \author{Author 1 \\ Address line \\  ... \\ Address line
%         \And  ... \And
%         Author n \\ Address line \\ ... \\ Address line}
% To start a separate ``row'' of authors use \AND, as in
% \author{Author 1 \\ Address line \\  ... \\ Address line
%         \AND
%         Author 2 \\ Address line \\ ... \\ Address line \And
%         Author 3 \\ Address line \\ ... \\ Address line}

\author{
  Miroojin Bakshi \\
  \texttt{miroojinb@iisc.ac.in} \\
  \And
  Saksham Rastogi \\
  \texttt{iitdsaksham@gmail.com} \\
  \And
  Danish Pruthi \\
  \texttt{danishp@iisc.ac.in} \\
}

%\author{
%  \textbf{First Author\textsuperscript{1}},
%  \textbf{Second Author\textsuperscript{1,2}},
%  \textbf{Third T. Author\textsuperscript{1}},
%  \textbf{Fourth Author\textsuperscript{1}},
%\\
%  \textbf{Fifth Author\textsuperscript{1,2}},
%  \textbf{Sixth Author\textsuperscript{1}},
%  \textbf{Seventh Author\textsuperscript{1}},
%  \textbf{Eighth Author \textsuperscript{1,2,3,4}},
%\\
%  \textbf{Ninth Author\textsuperscript{1}},
%  \textbf{Tenth Author\textsuperscript{1}},
%  \textbf{Eleventh E. Author\textsuperscript{1,2,3,4,5}},
%  \textbf{Twelfth Author\textsuperscript{1}},
%\\
%  \textbf{Thirteenth Author\textsuperscript{3}},
%  \textbf{Fourteenth F. Author\textsuperscript{2,4}},
%  \textbf{Fifteenth Author\textsuperscript{1}},
%  \textbf{Sixteenth Author\textsuperscript{1}},
%\\
%  \textbf{Seventeenth S. Author\textsuperscript{4,5}},
%  \textbf{Eighteenth Author\textsuperscript{3,4}},
%  \textbf{Nineteenth N. Author\textsuperscript{2,5}},
%  \textbf{Twentieth Author\textsuperscript{1}}
%\\
%\\
%  \textsuperscript{1}Affiliation 1,
%  \textsuperscript{2}Affiliation 2,
%  \textsuperscript{3}Affiliation 3,
%  \textsuperscript{4}Affiliation 4,
%  \textsuperscript{5}Affiliation 5
%\\
%  \small{
%    \textbf{Correspondence:} \href{mailto:email@domain}{email@domain}
%  }
%}

\begin{document}
\maketitle

%%% Abstract %%%
\begin{abstract}
With the rise of human-like text generation by large language models (LLMs), reliably attributing machine-generated content has become increasingly important for combating misinformation, ensuring content provenance, and enforcing responsible AI usage. Watermarking, which embeds identifiable statistical signals in the generated text, offers a promising solution. However, existing watermarking methods typically alter the sampling process during generation, making them vulnerable in open-weight settings where users have unrestricted control over how text is produced. Existing open-weight watermarking methods either require costly tuning strategies, exhibit weak detection strength, or can be trivially removed without affecting model performance. In this work, we propose a method that directly modifies the unembedding layer through a structured perturbation conditioned on the model's hidden states, in order to steer generation toward watermarked outputs. Our approach is training-free and integrates much faster than tuning-based watermarking methods. We conduct experiments demonstrating improved watermark detectability compared to existing approaches, without degrading text quality. To further assess practical utility, we evaluate the method's robustness to paraphrasing attacks, durability under post-hoc finetuning, and performance across a range of downstream tasks.
\end{abstract}P



%%% Introduction %%%
\section{Introduction}

As open-source large language models (LLMs) become increasingly capable and widely available, ensuring reliable attribution of AI-generated content has become more urgent
\sr{not clear why capable and widely available models require reliable attribution,
add a line of why due to issues like xxx we need reliable attribution}
. Watermarking—embedding imperceptible statistical signals into generated text—
\sr{do not need the erm dash here.}
has emerged as a promising technique to trace provenance and deter misuse. Traditional watermarking methods, however, rely on generation-time interventions that modify the decoding process to guide token selection \cite{kirchenbauer2023watermark}. These techniques are fundamentally incompatible with open-weight settings, where end-users have full access to and control over the model internals, including the sampling strategy.

To address this limitation, recent work embeds watermarks directly into a model's weights so it naturally generates watermarked text. Methods include distillation, which fine-tunes the model on outputs from a decoder-based watermarked teacher \cite{gu2023learnability}; reinforcement learning, using a separate detector model to assign rewards that guide watermark insertion \cite{xu2024learningwatermarkllmgeneratedtext}; and LoRA-based fine-tuning \cite{hu2022lora}, in which lightweight modules and a detector are trained jointly to encourage both watermark generation and detection \cite{elhassan2025can}. While effective, these approaches are often computationally expensive.

Edit-based watermarking methods 
\sr{we need to define this term before using it}
offer a lightweight alternative by directly modifying select model weights without any retraining. For instance, \textsc{Provably Robust} \cite{christ2024provably} introduces a Gaussian watermark into a newly added bias vector in the output layer. 
\sr{not sure if this a component added, since its there but just set to zero, let's focus on the that it simulates fixed list - not safe and easier to remove?}
While effective, this added component is non-standard and can be easily stripped without affecting model behavior. \textsc{Gaussmark} \cite{block2025gaussmark}, in contrast, avoids architectural changes by perturbing existing weight subsets and detects watermarks using a z-score computed from the dot product between the perturbation and the gradient of log-likelihood. However, a limitation of \textsc{Gaussmark} is that the detection signal is weak and the method requires both a forward pass and a partial backward pass, making it less practical for deployment.

In this work, we introduce a new
\sr{i would avoid claiming new}
\emph{edit-based watermarking framework} that modifies the unembedding layer weights—the parameters that project final hidden states to output vocabulary logits—by adding structured perturbations. These perturbations induce dynamic logit biases during generation that subtly influence token sampling in a detectable way. Unlike prior logit-based watermarking strategies such as \citet{kirchenbauer2023watermark} and \citet{liu2024adaptive}, which inject watermarks during decoding, our approach directly embeds the biasing logic into the model's weights. This design ensures that the watermark cannot be easily removed or circumvented, even when users have full control over the decoding process or inference code. Moreover, our framework is general-purpose: it supports multiple watermarking instantiations that produce different biasing patterns, all of which are compatible with a unified detection method.

To detect the watermark, we apply a simple and scalable test: the \textbf{average log-likelihood ratio (LLR)} per token between the watermarked model and a reference model.
\sr{we should unwatermarked model unless the reference model can be something entirely different}
This test captures subtle but consistent shifts in token probabilities introduced by the watermark. Crucially, it only requires forward passes and works across all watermarking strategies instantiated under our framework, provided they produce detectable logit perturbations.

We evaluate our method on three popular open-source LLMs, Llama-2-7b, Mistral-7B-v0.3 and Qwen2.5-3B under a range of conditions. Our results demonstrate that the proposed approach consistently achieves:

\begin{enumerate}

    \item Strong watermark detectability using the average LLR test

    \item Minimal impact on generation quality

    \item Robustness to paraphrasing and fine-tuning.

\end{enumerate}

The rest of the paper is organized as follows: Section~\ref{sec:background_related} provides background on language model watermarking, with a focus on the unique challenges posed by open-source settings. Section~\ref{sec:methodology} introduces a general unembedding-layer watermarking framework, along with two concrete instantiations and a corresponding statistical detection method. Section~\ref{sec:experiments} presents empirical evaluations of watermark detectability and text quality, including downstream task performance. Section~\ref{sec:paraphasing} examines robustness to paraphrasing attacks, and Section~\ref{sec:finetuning} investigates the durability of the watermark under model fine-tuning. Section~\ref{sec:conclusion} summarizes our findings and outlines directions for future work.

%%% Background and related Work %%%
\section{Background and Related Work}
\label{sec:background_related}

\subsection{Watermarking in Language Models}

Watermarking techniques aim to embed subtle signals into model-generated text that are imperceptible to humans but detectable through statistical analysis. Most existing approaches operate at decoding time, modifying next-token probabilities based on pseudorandom functions that assign scores or preferences to tokens in the vocabulary. A common strategy is to define a favored subset of tokens—the green list—by hashing the previous \(k\) tokens, and then bias generation toward these tokens using additive logit shifts or sampling-based constraints. For example, \citet{kirchenbauer2023watermark} apply soft logit biasing toward green-listed tokens, while \citet{aaronson2023reform} and \citet{kuditipudi2023robust} employ cryptographically driven scores to guide token sampling. Detection is typically performed through statistical tests that evaluate whether the distribution of generated tokens aligns with the expected green list patterns.

These methods typically involve a trade-off between two competing objectives: text generation quality and detection strength. Strong watermarking configurations—such as those using aggressive logit biasing or restrictive sampling—are easier to detect but tend to degrade text quality. In contrast, low-distortion watermarking setups preserve output quality but often result in signals that are harder to detect reliably.

\subsection{Open-Source Model Challenges}

In open-source settings, where users have full access to and control over the model, decoding-time watermarking becomes ineffective since the decoding logic can be easily bypassed. This has led to increasing interest in weight-based watermarking strategies that embed signals directly into the model's parameters.

Recent work explores different ways of modifying model weights to embed watermarks. One approach is watermark distillation \citep{gu2023learnability}, where a student model is trained to reproduce outputs from a decoding-time-watermarked teacher model. While this strategy can transfer the watermark, it is sensitive to post-hoc fine-tuning, and learning low-distortion watermarks requires a lot of training data. Another approach uses reinforcement learning to embed watermarks \citep{xu2024learningwatermarkllmgeneratedtext}. In this work, a detector model, trained jointly with the LLM, acts as a reward model in the reinforcement learning objective, steering the LLM to produce watermarked text while preserving fluency and utility. While this approach offers some benefits over distillation, it requires substantial computational resources and involves complex policy-based training objectives. \citet{elhassan2025can} eliminates the need for explicit reward modeling by directly incorporating a fully-differentiable detection objective into the training loss and uses low-rank adaptation \citet{hu2022lora} for efficient training. Despite its benefits, the approach relies on a min-max optimization that can be difficult to train stably.

Other approaches modify weights without retraining. \citet{christ2024provably} introduce a method that adds a Gaussian perturbation to the final-layer bias vector, inducing subtle but consistent shifts in token logits. These biases accumulate across generated tokens, enabling watermark detection. However, the approach relies on final-layer biases—which many LLMs omit—and the watermark can be easily erased by removing the bias. \citet{block2025gaussmark} propose injecting Gaussian perturbations into decoder weights, with detection based on computing the dot product between the noise vector and the gradient of the log-likelihood. This method requires both forward and backward passes and is highly sensitive to noise placement and magnitude.

\paragraph{Durability to Model Modifications.}
Durability is a central challenge in watermarking open-source models, which are frequently modified through quantization, pruning, merging, or fine-tuning. \citet{gloaguen2025towards} systematically evaluate existing approaches and find that none remain consistently detectable under such modifications.  In particular, distillation-based methods suffer from watermark decay under light supervision—even a few hundred steps of fine-tuning can erase the signal. Weight-editing schemes, while training-free, are often vulnerable to parameter shifts introduced by quantization or model merging. Their findings highlight the need for watermarking techniques designed explicitly with durability in mind.

\paragraph{Impact on downstream performance.}
Text quality is not the only metric affected by watermarking—\citet{ajith-etal-2024-downstream} show that even moderate-strength watermarks can significantly degrade performance on downstream tasks such as classification, QA, and generation. Since open-source watermarking methods embed the signal directly into model weights and cannot be easily undone once released, it is especially important to ensure that downstream task performance remains unaffected.

%%% Methodology %%%
\section{Methodology}
\label{sec:methodology}

This section presents our framework for watermarking language model outputs. We begin by formalizing the model setup and notation. We then describe the general mechanism for inducing logit-level watermark signals through linear operations on hidden states and specify the desired properties of such perturbations. Finally, we provide two illustrative instantiations that demonstrate how this framework can be instantiated in practice.

\subsection{Preliminaries}

Let a causal language model process a sequence of tokens drawn from a vocabulary \( \mathcal{V} = \{w_1, w_2, \dots, w_{|\mathcal{V}|}\} \). Let \( x_i \in \mathcal{V} \) denote the \( i \)-th token in the sequence, and let \( x_{<t} = (x_1, \dots, x_{t-1}) \) denote the prefix up to timestep \( t \).

The model computes a hidden representation \( h_t = f(x_{<t}) \in \mathbb{R}^d \), where \( f: \mathcal{V}^* \rightarrow \mathbb{R}^d \) is the model's internal encoding function. The logit vector \( v_t \in \mathbb{R}^{|\mathcal{V}|} \) for predicting the next token is given by:
\begin{equation}
    v_t = U h_t,
\end{equation}
where \( U \in \mathbb{R}^{|\mathcal{V}| \times d} \) is the unembedding matrix.

\subsection{A Framework for Watermarking via Unembedding Perturbation}

We present a general framework for watermarking the outputs of language models by introducing hidden-state-dependent biases to the token logits by adding a structured perturbation to \( U \). Specifically, we define a perturbed unembedding matrix:
\begin{equation}
    \tilde{U} = U + \Delta W,
\end{equation}
where \( \Delta W \in \mathbb{R}^{|\mathcal{V}| \times d} \) is the perturbation matrix. This results in a modified logit vector at timestep \( t \):
\begin{equation}
    \tilde{v}_t = \tilde{U} h_t = v_t + \Delta W h_t,
\end{equation}
The perturbation introduces an additive logit bias that varies with the internal state of the model and serves as a watermarking signal.

To be effective, the perturbation \( \Delta W \) should satisfy the following requirements:

\begin{itemize}
    \item \textbf{Detectable:} The induced logit biases should cause the generated text to accumulate identifiable signals that distinguish watermarked outputs from typical model generations. These signals should be reliably detectable using statistical or non-statistical methods applied to the generated text.


    \item \textbf{Controllable:} The perturbation should be parameterized to support explicit control over (i) the detectability or strength of the watermark signal, and (ii) the impact on text quality metrics such as fluency and perplexity. This enables a tunable tradeoff between visibility and stealth.

    \item \textbf{Logit variability:} The perturbation should induce logit biases that vary across timesteps and inputs. Fixed or invariant biases—such as those used in fixed green list schemes—are vulnerable to reverse engineering, as shown by \citet{rastogi2024revisitingrobustness}. In our framework, variability arises naturally since the logit perturbation is computed as a linear operation on the hidden state, which itself varies with context.

\end{itemize}

\paragraph{Norm-Based Analysis of Perturbation Strength.}
The total influence of \( \Delta W \) on the model's output logits can be analyzed using its Frobenius norm:
\[
    \|\Delta W\|_F = \left( \sum_{i=1}^{|\mathcal{V}|} \sum_{j=1}^{d} (\Delta W_{ij})^2 \right)^{1/2}.
\]
This norm provides a global upper bound on the magnitude of the logit perturbation at each timestep:
\[
    \|\Delta W h_t\| \leq \|\Delta W\|_F \cdot \|h_t\|.
\]

In practice, the norm \( \|h_t\| \) is typically stable and predictable in pretrained language models due to the widespread use of layer normalization \citep{ba2016layer}, which standardizes hidden activations across layers. As a result, we can approximate the bound using a model-dependent constant \( \mu_h \), the expected hidden state norm:
\[
    \|\Delta W h_t\| \lessapprox \|\Delta W\|_F \cdot \mu_h.
\]
While \( \mu_h \) may vary across architectures and layers, it is generally consistent within a given model.

The Frobenius norm serves as a useful and interpretable analytic quantity for characterizing the potential effect on model behavior. In contrast, \citet{block2025gaussmark} applies perturbations to arbitrary subsets of model weights, making their effect on output logits indirect and difficult to predict. As a result, tuning for detectability and output quality relies heavily on empirical calibration. In our framework, the perturbation is applied directly at the logit level, and its impact can be more deterministically understood by analyzing the norm of \( \Delta W \). Moreover, \( \|\Delta W\|_F \) provides a meaningful way to constrain the overall strength of the perturbation during the design or parameterization of \( \Delta W \).


\subsection{Illustrative Examples}
This framework accommodates a broad family of parameterizations for \( \Delta W \) that satisfy the above criteria. We now describe two example instantiations that demonstrate how the framework can be applied in practice.

\subsubsection{Green List Biasing}

This method draws inspiration from the watermarking approach of \citet{kirchenbauer2023watermark}, which boosts the logits of tokens belonging to a pseudorandomly selected \emph{green list}. At each timestep, the green list is generated using a pseudorandom function (PRF) that depends on a secret seed and the preceding \( n \) tokens in the prefix. This ensures that the token biases vary in a deterministic, structured, and context-sensitive manner.

To encode this behavior directly into the model weights, we express the perturbation matrix \( \Delta W \) as the product of two matrices:
\begin{equation}
    \Delta W = G H,
\end{equation}
where:

\begin{itemize}
    \item \( G \in \mathbb{R}^{|\mathcal{V}| \times C} \) contains \( C \) watermarking lists represented as row vectors. For each pseudo-class \( c \in \{1, \dots, C\} \), the corresponding row \( H_c \in \mathbb{R}^{|\mathcal{V}|} \) is defined as:
          \[
              (H_c)_i =
              \begin{cases}
                  \delta & \text{if } i \in \mathcal{G}_c \\
                  0      & \text{otherwise}
              \end{cases}
          \]
          where \( \delta > 0 \) is a fixed scalar that determines the strength of the watermark signal, and \( \mathcal{G}_c \subseteq \{1, \dots, |\mathcal{V}|\} \) is the green list for class \( c \), constructed using a pseudorandom function:
          \[
              \mathcal{G}_c = \left\{ i \in \{1, \dots, |\mathcal{V}|\} \;\middle|\; \mathrm{PRF}(\text{seed}, c, i) < \gamma \right\}.
          \]
          Here, \(\mathrm{PRF}(\cdot)\) is a keyed hash function seeded with a secret key, and \(\gamma \in (0,1)\) is a fixed threshold that determines the fraction of green-listed indices.

    \item \( H \in \mathbb{R}^{C \times d} \) maps each token's hidden state to a soft pseudo-class one-hot selector. To construct \( H \), we proceed as follows:
          \begin{enumerate}
              \item Run the base model on a large unlabeled corpus (e.g., OpenWebText) and collect hidden states from the last layer before the unembedding layer.
              \item Apply \( k \)-means clustering to these hidden states, assigning each to one of \( C \) pseudo-classes.
              \item Solve a ridge regression problem to map each hidden state \( h \in \mathbb{R}^d \) to a soft one-hot selector \( s \in \mathbb{R}^C \), approximating the discrete cluster assignments.
          \end{enumerate}
\end{itemize}

This construction ensures that hidden states in similar regions of representation space receive similar perturbations. Although the selectors in \( G \) are soft, they closely approximate discrete assignments and enable reliable watermark detection. The structured factorization of \( \Delta W \) into \( G \) and \( H \) allows multiple green list rules to be encoded in a single matrix perturbation.

\subsubsection{Gaussian Random Projection}

In this variant, the watermark is embedded by applying a fixed Gaussian perturbation to the unembedding matrix. Specifically, the perturbation matrix \( \Delta W \in \mathbb{R}^{|\mathcal{V}| \times d} \) is initialized as:
\begin{equation}
    \Delta W_{ij} \sim \mathcal{N}(0, 1).
\end{equation}

During generation, this produces a dynamic, input-dependent logit bias:
\[
    \Delta \ell_t = \Delta W h_t \in \mathbb{R}^{|\mathcal{V}|},
\]
which is added to the model’s original logits.

Unlike the green list biasing approach, which relies on discrete vocabulary partitions, this method introduces a continuous watermarking signal by projecting the hidden state through a random Gaussian matrix. Although the hidden state \( h_t \in \mathbb{R}^d \) is not necessarily Gaussian, the Central Limit Theorem ensures that the projected logits \( [\Delta \ell_t]_i = \langle \Delta W_i, h_t \rangle \) are approximately Gaussian-distributed when \( d \) is large:

\begin{theorem}[CLT for Logit Bias Projection]
    Let \( h_t \in \mathbb{R}^d \) be fixed, and let each row \( \Delta W_i \in \mathbb{R}^d \) be drawn independently from \( \mathcal{N}(0, I_d) \). Then the scalar projection
    \[
        [\Delta W h_t]_i = \langle \Delta W_i, h_t \rangle = \sum_{j=1}^d \Delta W_{ij} \cdot h_{tj}
    \]
    converges in distribution to \( \mathcal{N}(0, \|h_t\|^2) \) as \( d \to \infty \).
\end{theorem}

Consequently, the logit bias vector \( \Delta \ell_t \) is approximately distributed as \( \mathcal{N}(0, \|h_t\|^2 I) \), meaning that, at each timestep, about half of the tokens are boosted while the other half are suppressed. This introduces a subtle but statistically detectable watermark signal without degrading generation quality.

To ensure consistent perturbation magnitude across timesteps and model scales, we normalize the matrix using the expected norm of a hidden state vector and introduce a scaling hyperparameter \( \delta > 0 \) to control watermark strength:
\begin{equation}
    \Delta W \leftarrow \frac{\delta}{\mathbb{E}[\|h_t\|]} \cdot \Delta W.
\end{equation}

\subsection{Detection via Likelihood Ratio Test}

To detect the watermark, we use a length-normalized log-likelihood ratio (LLR) test between the watermarked and reference models:
\begin{equation} \label{eq:llr}
    \text{LLR}(x) = \frac{1}{T} \sum_{t=1}^{T} \log \frac{p_{\text{wm}}(x_t \mid x_{<t})}{p_{\text{ref}}(x_t \mid x_{<t})},
\end{equation}
where \( p_{\text{wm}} \) and \( p_{\text{ref}} \) denote the softmax probabilities computed using the watermarked and original unembedding matrices, respectively.

These probabilities are defined as:
\begin{align}
    p_{\text{ref}}(x_t \mid x_{<t}) & =
    \frac{\exp(U_{x_t} h_t)}{\sum_{j=1}^{|V|} \exp(U_j h_t)}, \\
    p_{\text{wm}}(x_t \mid x_{<t})  & =
    \frac{\exp(\tilde{U}_{x_t} h_t)}{\sum_{j=1}^{|V|} \exp(\tilde{U}_j h_t)},
\end{align}
where \( U \in \mathbb{R}^{|V| \times d} \) is the original unembedding matrix, \( \tilde{U} = U + \Delta W \) is the watermarked version, and \( h_t \in \mathbb{R}^d \) is the hidden state at timestep \( t \).

Substituting these into Equation~\ref{eq:llr}, we obtain:
\begin{equation}
    \begin{aligned}
        \text{LLR}(x)
         & = \frac{1}{T} \sum_{t=1}^{T} \Big(
        (\tilde{U}_{x_t} - U_{x_t}) h_t                                       \\
         & \quad - \log \frac{\sum_j e^{\tilde{U}_j h_t}}{\sum_j e^{U_j h_t}}
        \Big)
    \end{aligned}
\end{equation}

This expression decomposes the LLR into two interpretable terms:
\begin{itemize}
    \item A \textit{token-level logit shift} term, \( (\tilde{U}_{x_t} - U_{x_t}) h_t \), which directly reflects the effect of watermarking on the predicted token's logit.
    \item A \textit{partition function ratio} term, which captures the normalization difference across the entire vocabulary.
\end{itemize}

\paragraph{Detection Protocol.}
The model developer releases the model with the watermarked unembedding matrix \( \tilde{U} = U + \Delta W \), while keeping the original matrix \( U \)—and by extension the perturbation \( \Delta W \)—secret. To verify whether a given piece of text \( x \) was generated by the watermarked model, the developer computes the LLR between the public (watermarked) model and the hidden reference model using Equation~\ref{eq:llr}. A significantly positive LLR indicates the presence of the watermark signal.

This detection process can be kept private or deployed via a walled API, allowing third parties to query the watermark status of text without exposing the reference model or the perturbation matrix.


% TODO:
% - Describe expected value of LLR under the null hypothesis (no watermark) and the alternative hypothesis (watermark present).
% - Discuss why a LLR is better than logit difference. Show rigorousness justification.

%%% Experiments %%%
\section{Experiments and Results}
\label{sec:experiments}

% \subsection{Experimental Setup}
% \label{subsec:setup}
% We evaluate our watermarking strategies along four axes: \textbf{(i)}~detection accuracy under controlled distortion, \textbf{(ii)}~downstream task retention, \textbf{(iii)}~robustness to paraphrasing, and \textbf{(iv)}~durability against fine-tuning. All experiments are conducted on \texttt{meta-llama/Llama-2-13b-hf}. Additional results on Mistral and Qwen are provided in Appendix~\ref{sec:appendix:other-models}.

% \paragraph{Watermarked Sample Generation.} We generate 500 watermarked completions of 200 tokens each, sampled as continuations from 50-token prompts drawn from the \texttt{C4-realnewslike} corpus. The unwatermarked completions are generated from the same model without watermarking. Detection is performed on the continuation only, without access to the prompt, to simulate realistic post-hoc scenarios.

% \paragraph{Evaluation Metrics.} 
% \begin{itemize}
%     \item \textbf{Distortion:} Measured via perplexity (PPL) using a clean \texttt{Llama-2-13b-hf} model.
%     \item \textbf{Detection:} AUROC, Best F1, and TPR@FPR thresholds of 1\% and 0.1\%.
%     \item \textbf{Downstream utility:} Accuracy or EM on HellaSwag, GSM8k, and ARC-Challenge.
% \end{itemize}

% \paragraph{Baselines.} We compare both parameterizations of \(\delta W\) against the following baselines:
% \begin{itemize}
%     \item \textbf{GaussMark:} \(\sigma = 0.04\), applied to \texttt{model.layers.27.mlp.up\_proj.weight}.
%     \item \textbf{KGW Logit-Distilled:} \(\delta = 2.0\), \(k = 1\), \(\gamma = 0.25\).
%     \item \textbf{KGW Decoding (Greenlist):} \(k = 0\) as a high-detection upper bound.
% \end{itemize}
% All results are averaged over 3 random seeds.

% \subsection{Detection Accuracy vs. Distortion}
% \label{subsec:detection-vs-distortion}
% We analyze the trade-off between watermark strength and detectability by varying the hyperparameters used in constructing \(\delta W\). Key results are reported below.

% \begin{figure}[h]
%     \centering
%     \includegraphics[width=\linewidth]{figures/detection_vs_ppl.pdf}
%     \caption{Detection performance (TPR@1\%FPR, AUROC) vs. PPL for each method.}
%     \label{fig:detection-vs-ppl}
% \end{figure}

% \begin{table}[h]
%     \centering
%     \caption{Detection metrics (averaged over 3 seeds) at fixed PPL buckets.}
%     \label{tab:detection-metrics}
%     \begin{tabular}{lcccc}
%         \toprule
%         Method & PPL & AUROC & TPR@1\%FPR & TPR@0.1\%FPR \\
%         \midrule
%         Ours (\(\delta W\) A) & -- & -- & -- & -- \\
%         Ours (\(\delta W\) B) & -- & -- & -- & -- \\
%         GaussMark & -- & -- & -- & -- \\
%         KGW-Logit & -- & -- & -- & -- \\
%         KGW-Decoding & -- & -- & -- & -- \\
%         \bottomrule
%     \end{tabular}
% \end{table}

% \subsection{Impact on Downstream Task Accuracy}
% \label{subsec:downstream}
% We evaluate whether watermarking affects task accuracy. Each method is evaluated on the dev sets of three tasks:

% \begin{table}[h]
%     \centering
%     \caption{Downstream performance (accuracy or EM) on unwatermarked vs. watermarked models.}
%     \label{tab:downstream}
%     \begin{tabular}{lccc}
%         \toprule
%         Method & HellaSwag & GSM8k & ARC-Challenge \\
%         \midrule
%         No watermark & -- & -- & -- \\
%         Ours (\(\delta W\) A) & -- & -- & -- \\
%         Ours (\(\delta W\) B) & -- & -- & -- \\
%         GaussMark & -- & -- & -- \\
%         KGW-Logit & -- & -- & -- \\
%         \bottomrule
%     \end{tabular}
% \end{table}

% \subsection{Robustness to Paraphrasing}
% \label{subsec:paraphrasing}
% To assess robustness, we paraphrase watermarked samples using DIPPER \citep{krishna2023paraphrasing} at lexical diversity levels of 20 and 60. Detection metrics are reported below.

% \begin{figure}[h]
%     \centering
%     \includegraphics[width=\linewidth]{figures/paraphrase_robustness.pdf}
%     \caption{Detection performance under paraphrasing.}
%     \label{fig:paraphrasing}
% \end{figure}

% \begin{table}[h]
%     \centering
%     \caption{Detection metrics under paraphrasing at diversity 20/60.}
%     \label{tab:paraphrase-results}
%     \begin{tabular}{lcccc}
%         \toprule
%         Method & LexDiv & AUROC & Best F1 & TPR@1\%FPR \\
%         \midrule
%         Ours (\(\delta W\) A) & 20 & -- & -- & -- \\
%         Ours (\(\delta W\) A) & 60 & -- & -- & -- \\
%         GaussMark & 20 & -- & -- & -- \\
%         KGW-Decoding & 60 & -- & -- & -- \\
%         \bottomrule
%     \end{tabular}
% \end{table}

% \subsection{Durability under Fine-Tuning}
% \label{subsec:durability}
% We simulate a post-deployment adversary performing LoRA-based fine-tuning over 2500 steps on 1\% of OpenWebText.

% \paragraph{Fine-Tuning Configuration.}
% \begin{itemize}
%     \item \textbf{Optimizer:} AdamW, LR = 1e-5, cosine scheduler, warmup = 500 steps
%     \item \textbf{Batch size:} 32, Sequence length: 512
%     \item \textbf{LoRA:} \(r=8\), \(\alpha=32\), dropout = 0.05
% \end{itemize}

% \paragraph{Target Modules.}
% \begin{itemize}
%     \item \textbf{Ours:} \texttt{lm\_head} (unembedding)
%     \item \textbf{Baselines:} \texttt{mlp.\{up,down,gate\}_proj} in top 10 layers + \texttt{lm\_head}
% \end{itemize}

% \begin{figure}[h]
%     \centering
%     \includegraphics[width=\linewidth]{figures/durability_over_ft.pdf}
%     \caption{Detection degradation curve over fine-tuning steps.}
%     \label{fig:durability}
% \end{figure}

% \begin{table}[h]
%     \centering
%     \caption{Detection metric drop (AUROC, TPR@1\%FPR) every 500 steps of fine-tuning.}
%     \label{tab:durability}
%     \begin{tabular}{lcccccc}
%         \toprule
%         Method & Step 0 & Step 500 & Step 1000 & Step 1500 & Step 2000 & Step 2500 \\
%         \midrule
%         Ours (\(\delta W\) A) & -- & -- & -- & -- & -- & -- \\
%         GaussMark & -- & -- & -- & -- & -- & -- \\
%         KGW-Logit & -- & -- & -- & -- & -- & -- \\
%         \bottomrule
%     \end{tabular}
% \end{table}


%%% Robustness to Paraphrasing %%%
\input{sections/paraphrasing.tex}


%%% Durability against Fine-tuning %%%
\input{sections/finetuning.tex}


%%% Conclusion %%%
\input{sections/conclusion.tex}

\section{Preamble}

The first line of the file must be
\begin{quote}
\begin{verbatim}
\documentclass[11pt]{article}
\end{verbatim}
\end{quote}

To load the style file in the review version:
\begin{quote}
\begin{verbatim}
\usepackage[review]{acl}
\end{verbatim}
\end{quote}
For the final version, omit the \verb|review| option:
\begin{quote}
\begin{verbatim}
\usepackage{acl}
\end{verbatim}
\end{quote}

To use Times Roman, put the following in the preamble:
\begin{quote}
\begin{verbatim}
\usepackage{times}
\end{verbatim}
\end{quote}
(Alternatives like txfonts or newtx are also acceptable.)

Please see the \LaTeX{} source of this document for comments on other packages that may be useful.

Set the title and author using \verb|\title| and \verb|\author|. Within the author list, format multiple authors using \verb|\and| and \verb|\And| and \verb|\AND|; please see the \LaTeX{} source for examples.

By default, the box containing the title and author names is set to the minimum of 5 cm. If you need more space, include the following in the preamble:
\begin{quote}
\begin{verbatim}
\setlength\titlebox{<dim>}
\end{verbatim}
\end{quote}
where \verb|<dim>| is replaced with a length. Do not set this length smaller than 5 cm.

\section{Document Body}

\subsection{Footnotes}

Footnotes are inserted with the \verb|\footnote| command.\footnote{This is a footnote.}

\subsection{Tables and figures}

See Table~\ref{tab:accents} for an example of a table and its caption.
\textbf{Do not override the default caption sizes.}

\begin{table}
  \centering
  \begin{tabular}{lc}
    \hline
    \textbf{Command} & \textbf{Output} \\
    \hline
    \verb|{\"a}|     & {\"a}           \\
    \verb|{\^e}|     & {\^e}           \\
    \verb|{\`i}|     & {\`i}           \\
    \verb|{\.I}|     & {\.I}           \\
    \verb|{\o}|      & {\o}            \\
    \verb|{\'u}|     & {\'u}           \\
    \verb|{\aa}|     & {\aa}           \\\hline
  \end{tabular}
  \begin{tabular}{lc}
    \hline
    \textbf{Command} & \textbf{Output} \\
    \hline
    \verb|{\c c}|    & {\c c}          \\
    \verb|{\u g}|    & {\u g}          \\
    \verb|{\l}|      & {\l}            \\
    \verb|{\~n}|     & {\~n}           \\
    \verb|{\H o}|    & {\H o}          \\
    \verb|{\v r}|    & {\v r}          \\
    \verb|{\ss}|     & {\ss}           \\
    \hline
  \end{tabular}
  \caption{Example commands for accented characters, to be used in, \emph{e.g.}, Bib\TeX{} entries.}
  \label{tab:accents}
\end{table}

As much as possible, fonts in figures should conform
to the document fonts. See Figure~\ref{fig:experiments} for an example of a figure and its caption.

Using the \verb|graphicx| package graphics files can be included within figure
environment at an appropriate point within the text.
The \verb|graphicx| package supports various optional arguments to control the
appearance of the figure.
You must include it explicitly in the \LaTeX{} preamble (after the
\verb|\documentclass| declaration and before \verb|\begin{document}|) using
\verb|\usepackage{graphicx}|.

\begin{figure}[t]
  \includegraphics[width=\columnwidth]{example-image-golden}
  \caption{A figure with a caption that runs for more than one line.
    Example image is usually available through the \texttt{mwe} package
    without even mentioning it in the preamble.}
  \label{fig:experiments}
\end{figure}

\begin{figure*}[t]
  \includegraphics[width=0.48\linewidth]{example-image-a} \hfill
  \includegraphics[width=0.48\linewidth]{example-image-b}
  \caption {A minimal working example to demonstrate how to place
    two images side-by-side.}
\end{figure*}

\subsection{Hyperlinks}

Users of older versions of \LaTeX{} may encounter the following error during compilation:
\begin{quote}
\verb|\pdfendlink| ended up in different nesting level than \verb|\pdfstartlink|.
\end{quote}
This happens when pdf\LaTeX{} is used and a citation splits across a page boundary. The best way to fix this is to upgrade \LaTeX{} to 2018-12-01 or later.

\subsection{Citations}

\begin{table*}
  \centering
  \begin{tabular}{lll}
    \hline
    \textbf{Output}           & \textbf{natbib command} & \textbf{ACL only command} \\
    \hline
    \citep{Gusfield:97}       & \verb|\citep|           &                           \\
    \citealp{Gusfield:97}     & \verb|\citealp|         &                           \\
    \citet{Gusfield:97}       & \verb|\citet|           &                           \\
    \citeyearpar{Gusfield:97} & \verb|\citeyearpar|     &                           \\
    \citeposs{Gusfield:97}    &                         & \verb|\citeposs|          \\
    \hline
  \end{tabular}
  \caption{\label{citation-guide}
    Citation commands supported by the style file.
    The style is based on the natbib package and supports all natbib citation commands.
    It also supports commands defined in previous ACL style files for compatibility.
  }
\end{table*}

Table~\ref{citation-guide} shows the syntax supported by the style files.
We encourage you to use the natbib styles.
You can use the command \verb|\citet| (cite in text) to get ``author (year)'' citations, like this citation to a paper by \citet{Gusfield:97}.
You can use the command \verb|\citep| (cite in parentheses) to get ``(author, year)'' citations \citep{Gusfield:97}.
You can use the command \verb|\citealp| (alternative cite without parentheses) to get ``author, year'' citations, which is useful for using citations within parentheses (e.g. \citealp{Gusfield:97}).

A possessive citation can be made with the command \verb|\citeposs|.
This is not a standard natbib command, so it is generally not compatible
with other style files.

\subsection{References}


The \LaTeX{} and Bib\TeX{} style files provided roughly follow the American Psychological Association format.
If your own bib file is named \texttt{custom.bib}, then placing the following before any appendices in your \LaTeX{} file will generate the references section for you:
\begin{quote}
\begin{verbatim}
\bibliography{custom}
\end{verbatim}
\end{quote}

You can obtain the complete ACL Anthology as a Bib\TeX{} file from \url{https://aclweb.org/anthology/anthology.bib.gz}.
To include both the Anthology and your own .bib file, use the following instead of the above.
\begin{quote}
\begin{verbatim}
\bibliography{anthology,custom}
\end{verbatim}
\end{quote}

Please see Section~\ref{sec:bibtex} for information on preparing Bib\TeX{} files.

\subsection{Equations}

An example equation is shown below:
\begin{equation}
  \label{eq:example}
  A = \pi r^2
\end{equation}

Labels for equation numbers, sections, subsections, figures and tables
are all defined with the \verb|\label{label}| command and cross references
to them are made with the \verb|\ref{label}| command.

This an example cross-reference to Equation~\ref{eq:example}.

\subsection{Appendices}

Use \verb|\appendix| before any appendix section to switch the section numbering over to letters. See Appendix~\ref{sec:appendix} for an example.

\section{Bib\TeX{} Files}
\label{sec:bibtex}

Unicode cannot be used in Bib\TeX{} entries, and some ways of typing special characters can disrupt Bib\TeX's alphabetization. The recommended way of typing special characters is shown in Table~\ref{tab:accents}.

Please ensure that Bib\TeX{} records contain DOIs or URLs when possible, and for all the ACL materials that you reference.
Use the \verb|doi| field for DOIs and the \verb|url| field for URLs.
If a Bib\TeX{} entry has a URL or DOI field, the paper title in the references section will appear as a hyperlink to the paper, using the hyperref \LaTeX{} package.

\section*{Limitations}

Since December 2023, a "Limitations" section has been required for all papers submitted to ACL Rolling Review (ARR). This section should be placed at the end of the paper, before the references. The "Limitations" section (along with, optionally, a section for ethical considerations) may be up to one page and will not count toward the final page limit. Note that these files may be used by venues that do not rely on ARR so it is recommended to verify the requirement of a "Limitations" section and other criteria with the venue in question.

\section*{Acknowledgments}

This document has been adapted
by Steven Bethard, Ryan Cotterell and Rui Yan
from the instructions for earlier ACL and NAACL proceedings, including those for
ACL 2019 by Douwe Kiela and Ivan Vuli\'{c},
NAACL 2019 by Stephanie Lukin and Alla Roskovskaya,
ACL 2018 by Shay Cohen, Kevin Gimpel, and Wei Lu,
NAACL 2018 by Margaret Mitchell and Stephanie Lukin,
Bib\TeX{} suggestions for (NA)ACL 2017/2018 from Jason Eisner,
ACL 2017 by Dan Gildea and Min-Yen Kan,
NAACL 2017 by Margaret Mitchell,
ACL 2012 by Maggie Li and Michael White,
ACL 2010 by Jing-Shin Chang and Philipp Koehn,
ACL 2008 by Johanna D. Moore, Simone Teufel, James Allan, and Sadaoki Furui,
ACL 2005 by Hwee Tou Ng and Kemal Oflazer,
ACL 2002 by Eugene Charniak and Dekang Lin,
and earlier ACL and EACL formats written by several people, including
John Chen, Henry S. Thompson and Donald Walker.
Additional elements were taken from the formatting instructions of the \emph{International Joint Conference on Artificial Intelligence} and the \emph{Conference on Computer Vision and Pattern Recognition}.

% Bibliography entries for the entire Anthology, followed by custom entries
%\bibliography{anthology,custom}
% Custom bibliography entries only
\bibliography{custom}

\appendix

\section{Example Appendix}
\label{sec:appendix}

This is an appendix.

\end{document}